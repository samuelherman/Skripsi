\chapter{Pendahuluan}
\label{chap:pendahuluan}

\section{Latar Belakang}
\label{sec:latarbelakang}

Saat ini jumlah dosen dan jumlah mahasiswa menjadi pemasalahan, disebabkan
minimnya jumlah dosen. Kurangnya tenaga dosen mengakibatkan seorang dosen harus
menjadi dosen wali atau dosen pembimbing banyak mahasiswa dalam satu waktu.
Kesulitan yang dimiliki oleh setiap dosen adalah kesulitan dalam mengingat
perkembangan setiap mahasiswa serta sejarah setiap mahasiswa.

Maka dari itu berdasarkan jabaran masalah diatas, baik untuk dibuat sebuah
perangkat lunak yang mencatat riwayat setiap mahasiswa. Dimana semua dosen yang
telah terautentikasi dapat berkontribusi untuk memantau perkembangan setiap anak
walinya. Lalu setiap aksi yang dilakukan pada mahasiswa baik aksi {\it edit}
maupun aksi {\it view} dicatat sehingga dapat dilihat historinya. Dan yang
terakhir setiap perubahan dicatat revisinya sehingga dapat dipantau apa saja
yang telah dirubah.

Untuk membangun aplikasi tersebut, teknologi yang digunakan adalah Google
Authentication, Markdown Syntax, StrapdownJS, Zurb Foundation, PHP, dan MySQL.
Google Authentication akan digunakan untuk mengauthentikasi setiap dosen pada
saat login. Kemudian semua format penulisan akan menggunakan Markdown Syntax.
Lalu untuk menampilkan penulisan dalam fotmat Markdown Syntax ke halaman website
menggunakan StrapdownJS. Perangkat lunak ini menggunakan Zurb Foundation untuk
membuat tampilan antarmuka. Yang terakhir untuk kebutuhan fungsional dan basis
data akan menggunakan PHP dan MySQL.

\section{Rumusan Masalah}
Berdasarkan latar belakang maka dapat dirumuskan permasalahan sebagai berikut:
\begin{itemize}
	\item Bagaimana mengautentikasi pengguna menggunakan Google Authentication?
	\item Bagaimana menggunakan teks dengan format Markdown?
	\item Bagaimana menampilkan teks dengan format Markdown ke halaman website?
	\item Bagaimana merancang antarmuka Sistem Informasi Riwayat Mahasiswa
	menggunakan Zurb Foundation?
	\item Bagaimana mengimplementasikan Sistem Informasi Riwayat Mahasiswa yang
	telah dirancang kedalam {\it script} PHP?
\end{itemize}

\section{Tujuan}
Berdasarkan rumusan masalah yang ditulis dalam sub bab 2, tujuan utama yang
ingin dicapai melalui penelitian ini adalah:
\begin{itemize}
	\item Mengautentikasi pengguna menggunakan Google Authentication.
	\item Menggunakan teks dengan format Markdown Syntax.
	\item Menampilkan teks dengan format Markdown Syntax ke halaman website.
	\item Merancang antarmuka Sistem Informasi Riwayat Mahasiswa menggunakan Zurb
	Foundation.
	\item Mengimplementasikan Sistem Informasi Riwayat Mahasiswa yang telah
	dirancang kedalam {\it script} PHP.
\end{itemize}

\section{Batasan Masalah}
Dalam penelitian ini ditetapkan batasan-batasan yang akan menjadi pedoman dalam
pelaksanaan penelitian:
\begin{itemize}
	\item Perangkat lunak akan memiliki 6 fitur yaitu: Login, Pilih mahasiswa,
	Melihat info mahasiswa, Edit mahasiswa, Lihat histori, dan Membuat entri baru.
	\item Untuk fitur login hanya untuk dosen yang diakhiri dengan @unpar.ac.id dan
	{\it username} bukan angka semua.
	\item Untuk fitur pilih mahasiswa, pengguna dapat memilih mahasiswa yang ingin
	dilihat atau dirubah dan pengguna juga bisa menekan tombol "Add" untuk menambah
	mahasiswa baru.
	\item Untuk fitur melihat info mahasiswa, pengguna dapat melihat info terkini
	dari mahasiswa dan aksi ini dicatat dalam log untuk alasan penjagaan privasi.
	\item Untuk fitur edit mahasiswa, pengguna dapat mengubah info mahasiswa dan
	aksi ini juga dicatat dalam log.
	\item Untuk fitur lihat histori, pengguna dapat melihat histori setiap aksi
	perubahan atau aksi {\it view}.
	\item Untuk fitur membuat entri baru, saat membuat entri baru akan dibuatkan
	{\it template} sehingga kedepannya isi info setiap mahasiswa seragam.
\end{itemize}

\section{Metodologi Penelitian}
Metodologi yang digunakan untuk menyusun penelitian:
\begin{itemize}
	\item Melakukan studi pustaka mengenai teknologi yang akan digunakan untuk
	membangun Sistem Informasi Riwayat Mahasiswa.
	\item Menganalisis cara kerja teknologi yang akan digunakan untuk membangun
	Sistem Informasi Riwayat Mahasiswa.
	\item Merancang Sistem Informasi Riwayat Mahasiswa yang akan dibuat.
	\item Melakukan implementasi untuk Sistem Informasi Riwayat Mahasiswa yang
	telah dirancang ke dalam PHP.
	\item Melakukan pengujian perangkat lunak yang telah diimplementasikan.
\end{itemize}

\section{Sistematika Pembahasan}
Sistematika pembahasan dalam penelitian ini adalah sebagai berikut:
\begin{itemize}
	\item Bab I Pendahuluan\\
	Bab ini menjelaskan latar belakang permasalahan, rumusan masalah, tujuan,
	batasan masalah, metodologi penelitian, dan sistematika pembahasan.
	\item Bab II Dasar Teori\\
	Bab ini menjelaskan teori-teori dasar mengenai Google Authentication, Markdown
	Syntax, StrapdownJS, Zurb Foundation, PHP, dan MySQL yang menjadi refrensi
	utama dalam pelaksanaan penelitian.
	\item Bab III Analisis\\
	Bab ini berisi analisis mengenai Google Authentication, Markdown Syntax,
	StrapdownJS, Zurb Foundation, PHP, dan MySQL yang akan digunakan pada
	penelitian ini.
	\item Bab IV Perancangan\\
	Bab ini berisi perancangan Sistem Informasi Riwayat Mahasiswa yang akan dibuat.
	\item Bab V Implementasi dan Pengujian\\
	Bab ini berisi pengimplementasian dan pengujian Sistem Informasi Riwayat
	Mahasiswa.
	\item Bab VI Kesimpulan dan Saran\\
	Bab ini berisi kesimpulan dari hasil penelitian dan saran untuk pengembangan
	lebih lanjut.
\end{itemize}