\chapter{Kesimpulan dan Saran}
\label{chap:kesimpulandansaran}

Pada bab ini akan diberikan kesimpulan yang didapat dari proses perancangan dan
pengujian perangkat lunak yang dibangun, juga saran-saran untuk penelitian ini
jika ingin dikembangkan di kemudian hari.

\section{Kesimpulan}
\label{sec:kesimpulan}
Kesimpulan yang didapat dari pembangunan perangkat lunak Sistem Inforamasi
Riwayat Mahasiswa (SIRM) anatara lain :
\begin{enumerate}[(1)]
  \item Penggunaan Google Authentication berhasil mengautentikasi pengguna dan
  berfungsi dengan baik pada perangkat lunak Sistem Informasi Riwayat Mahasiswa
  (SIRM).
  \item Penggunaan Markdown berhasil membuat format penulisan seragam
  dan berfungsi dengan baik pada perangkat lunak Sistem Informasi Riwayat
  Mahasiswa (SIRM).
  \item Penggunaan StrapdownJS berhasil menampilkan teks dengan sintaks Markdown
  ke halaman website dan berfungsi dengan baik pada perangkat lunak Sistem
  Informasi Riwayat Mahasiswa (SIRM).
  \item Penggunaan Zurb Foundation berhasil membuat tampilan antarmuka perangkat
  lunak Sistem Informasi Riwayat Mahasiswa (SIRM) menjadi responsif, tampilan
  antarmuka mengikuti lebar mesin pencari dan/atau layar komputer tanpa
  melakukan permintaan tambahan ke server.
  \item Pada tahap pengujian dapat disimpulkan bahwa perangkat lunak Sistem
  Informasi Riwayat Mahasiswa (SIRM) sudah berjalan dengan baik dan memberikan
  {\it output} sesuai yang diharapkan pengguna.
\end{enumerate}

\section{Saran}
\label{sec:saran}
Saran yang dapat diberikan untuk perbaikan dan pengembangan Sistem Informasi
Riwayat Mahasiswa (SIRM) antara lain :
\begin{enumerate}[(1)]
  \item Agar setiap pengguna dapat saling berinteraksi untuk pengembangan dapat
  ditambahkan fitur-fitur seperti kirim pesan, {\it chat-room}, dan forum.
  \item Pada fungsi histori untuk pengembangan dapat ditambahkan keterangan yang
  membandingkan versi baru dan versi lama. Jadi pengguna mengetahui bagian mana
  yang dihapus dan bagian mana yang ditambah atau dirubah.
\end{enumerate}