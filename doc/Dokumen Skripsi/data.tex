%_____________________________________________________________________________
%=============================================================================
% data.tex v1.1 (21-10-2012) \ldots dibuat oleh Lionov - Informatika FTIS UNPAR
%
% Perubahan pada versi 1.1 (06-08-2012)
% - Data dosen dipindah ke dosen.tex agar jika ada perubahan/update data dosen
%   mahasiswa tidak perlu mengubah data.tex
% - Perubahan pada keterangan dosen 
%
% Perubahan pada versi sebelumnya
% Versi 1.0 (06-08-2012)
% - Perubahan pada beberapa keterangan 
% versi 0.9b (09-07-2012):
% - Menambahkan data judul dalam bahasa inggris
% - Membuat bagian khusus untuk judul (bagian VIII)
% - Perbaikan pada gelar dosen
%_____________________________________________________________________________
%=============================================================================
% 								BAGIAN -
%=============================================================================
% Ini adalah file data (data.tex)
% Masukkan ke dalam file ini, data-data yang diperlukan oleh template ini
% Cara memasukkan data dijelaskan di setiap bagian
% Data yang WAJIB dan HARUS diisi dengan baik dan benar adalah SELURUHNYA !!
%=============================================================================
%_____________________________________________________________________________
%=============================================================================
% 								BAGIAN I
%=============================================================================
% Tambahkan package2 lain yang anda butuhkan di sini
%=============================================================================
\usepackage{booktabs}
\usepackage[table]{xcolor}
\usepackage{longtable}
\usepackage{amsmath}
\usepackage{enumerate}
\usepackage{pifont}
\usepackage{titlesec}
\setcounter{secnumdepth}{4}
\lstset{
    literate={~} {$\sim$}{1}
}

%\titleformat{\paragraph}
%{\normalfont\normalsize\bfseries}{\theparagraph}{1em}{}
%\titlespacing*{\paragraph}
%{0pt}{3.25ex plus 1ex minus .2ex}{1.5ex plus .2ex}
%=============================================================================

%_____________________________________________________________________________
%=============================================================================
% 								BAGIAN II
%=============================================================================
% Mode dokumen: menetukan halaman depan dari dokumen, apakah harus mengandung 
% prakata/pernyataan/abstrak dll (termasuk daftar gambar/tabel/isi) ?
% - kosong : tidak ada halaman depan sama sekali (untuk dokumen yang 
%            dipergunakan pada proses bimbingan)
% - cover : cover saja tanpa daftar isi, gambar dan tabel
% - sidang : cover, daftar isi, gambar, tabel (IT: UTS-UAS Seminar 
%			 dan UTS TA)
% - sidang_akhir : mode sidang + abstrak + abstract
% - final : seluruh halaman awal dokumen (untuk cetak final)
% Jika tidak ingin mencetak daftar tabel/gambar (misalkan karena tidak ada 
% isinya), edit manual di baris 439 dan 440 pada file main.tex
%=============================================================================
%\mode{kosong}
%\mode{cover}
\mode{sidang}
%\mode{sidang_akhir}
%\mode{final} 
%=============================================================================

%_____________________________________________________________________________
%=============================================================================
% 								BAGIAN III
%=============================================================================
% Line numbering: penomoran setiap baris, otomatis di-reset setiap berganti
% halaman
% - yes: setiap baris diberi nomor
% - no : baris tidak diberi nomor, otomatis untuk mode final
%=============================================================================
\linenumber{yes}
%=============================================================================

%_____________________________________________________________________________
%=============================================================================
% 								BAGIAN IV
%=============================================================================
% Linespacing: jarak antara baris 
% - single: opsi yang disediakan untuk bimbingan, jika pembimbing tidak
%            keberatan (untuk menghemat kertas)
% - onehalf: default dan wajib (dan otomatis) jika ingin mencetak dokumen
%            final/untuk sidang.
% - double : jarak yang lebih lebar lagi, jika pembimbing berniat memberi 
%            catatan yg banyak di antara baris (dianjurkan untuk bimbingan)
%=============================================================================
%\linespacing{single}
%\linespacing{onehalf}
\linespacing{double}
%=============================================================================

%_____________________________________________________________________________
%=============================================================================
% 								BAGIAN V
%=============================================================================
% Bab yang akan dicetak: isi dengan angka 1,2,3 s.d 9, sehingga bisa digunakan
% untuk mencetak hanya 1 atau beberapa bab saja
% Jika lebih dari 1 bab, pisahkan dengan ',', bab akan dicetak terurut sesuai 
% urutan bab.
% Untuk mencetak seluruh bab, kosongkan parameter (i.e. \bab{})  
% Catatan: Jika ingin menambahkan bab ke-10 dan seterusnya, harus dilakukan 
% secara manual
%=============================================================================
\bab{1,2,3,4,5}
%=============================================================================

%_____________________________________________________________________________
%=============================================================================
% 								BAGIAN VI
%=============================================================================
% Lampiran yang akan dicetak: isi dengan huruf A,B,C s.d I, sehingga bisa 
% digunakan untuk mencetak hanya 1 atau beberapa lampiran saja
% Jika lebih dari 1 lampiran, pisahkan dengan ',', lampiran akan dicetak 
% terurut sesuai urutan lampiran
% Jika tidak ingin mencetak lampiran apapun, kosongkan parameter 
% (i.e. \lampiran{}). Tidak ada opsi mencetak seluruh lampiran, anda harus
% menuliskan semuanya sendiri (i.e. \lampiran{A,B,C,D,E,F,G,H,I}) 
% Catatan: Jika ingin menambahkan lampiran ke-J dan seterusnya, harus 
% dilakukan secara manual
%=============================================================================
\lampiran{}
%=============================================================================

%_____________________________________________________________________________
%=============================================================================
% 								BAGIAN VII
%=============================================================================
% Data diri dan skripsi/tugas akhir
% - namanpm: Nama dan NPM anda, penggunaan huruf besar untuk nama harus benar
%			 dan gunakan 10 digit npm UNPAR, PASTIKAN BAHWA BENAR !!!
%			 (e.g. \namanpm{Jane Doe}{1992710001}
% - judul : Dalam bahasa Indonesia, perhatikan penggunaan huruf besar, judul
%			tidak menggunakan huruf besar seluruhnya !!! 
% - tanggal : isi dengan {tangga}{bulan}{tahun} dalam angka numerik, jangan 
%			  menuliskan kata (e.g. AGUSTUS) dalam isian bulan
%			  Tanggal ini adalah tanggal dimana anda akan melaksanakan sidang 
%			  ujian akhir skripsi/tugas akhir
% - pembimbing: isi dengan pembimbing anda, lihat daftar dosen di file dosen.tex
%				jika pembimbing hanya 1, kosongkan parameter kedua 
%				(e.g. \pembimbing{\JND}{  } )
% - penguji : isi dengan para penguji anda, lihat daftar dosen di file dosen.tex
%				(e.g. \penguji{\JHD}{\JCD} )
%=============================================================================
\namanpm{Samuel Herman}{2010730013}
\tanggal{10}{02}{2015}
\pembimbing{\PAS}{}     %Lihat singkatan pembimbing anda di file dosen.tex
\penguji{}{} 		%Lihat singkatan penguji anda di file dosen.tex
%=============================================================================

%_____________________________________________________________________________
%=============================================================================
% 								BAGIAN VIII
%=============================================================================
% Judul dan title : judul bhs indonesia dan inggris
% - judulINA: judul dalam bahasa indonesia
% - judulENG: title in english
% PERHATIAN: - langsung mulai setelah '{' awal, jangan mulai menulis di baris 
%			   bawahnya
%			 - Gunakan \texorpdfstring{\\}{} untuk pindah ke baris baru
%			 - Judul TIDAK ditulis dengan menggunakan huruf besar seluruhnya !!
%=============================================================================

\judulINA{Sistem Informasi Riwayat Mahasiswa}

\judulENG{Student Information History Recording System}

%_____________________________________________________________________________
%=============================================================================
% 								BAGIAN IX
%=============================================================================
% Abstrak dan abstract : abstrak bhs indonesia dan inggris
% - abstrakINA: abstrak bahasa indonesia
% - abstrakENG: abstract in english
% PERHATIAN: langsung mulai setelah '{' awal, jangan mulai menulis di baris 
%			 bawahnya
%=============================================================================

\abstrakINA{Penelitian ini akan mengembangkan CAPTCHA yang memiliki kualitas yang baik.
Kualitas yang baik akan didapat apabila CAPTCHA yang dikembangkan dapat diselesaikan oleh manusia dan tidak dapat diselesaikan oleh komputer.
Gambar CAPTCHA yang dibangkitkan akan menggunakan tiga efek distorsi untuk meningkatkan kualitas CAPTCHA.
Efek distorsi pertama adalah merotasi karakter yang akan diacak dari 0^{\circ}$ sampai 360^{\circ}$.
Efek distorsi kedua adalah membalik karakter secara horisontal maupun vertikal.
Dan efek distorsi yang terakhir adalah menambahan garis yang ditambahkan pada gambar secara acak.
Ketiga efek distorsi tersebut akan digabungkan sehingga menghasilkan CAPTCHA dengan kualitas yang baik.
}

\abstrakENG{This research will develop a CAPTCHA that has good quality.
Good quality will be obtained if the CAPTCHA which is developed can be solved by humans and cannot be solved by a computer.
A picture captcha raised will use of three effects of distortion to improve the quality of captcha.
The first distortion effects is a rotation of characters that will be scrambled from 0^{\circ}$ to 360^{\circ}$.
The second distortion effects to flip characters to the horizontal as well as vertical.
And the effects of distortion that last one is add a line that is added in figure at random.
The third effect of the distortion will be combined so as to generate a CAPTCHA with a good quality.
}

%=============================================================================

%_____________________________________________________________________________
%=============================================================================
% 								BAGIAN X
%=============================================================================
% Kata-kata kunci dan keywords : diletakkan di bawah abstrak (ina dan eng)
% - kunciINA: kata-kata kunci dalam bahasa indonesia
% - kunciENG: keywords in english
%=============================================================================
\kunciINA{CAPTCHA, Distorsi, Merotasi, Membalik, Menambah Garis, Kualitas Yang Baik}

\kunciENG{CAPTCHA, Distortion, Rotation, Flip, Add A Line, Good Quality}
%=============================================================================

%_____________________________________________________________________________
%=============================================================================
% 								BAGIAN XI
%=============================================================================
% Persembahan : kepada siapa anda mempersembahkan skripsi ini ...
%=============================================================================
\untuk{Dipersembahkan untuk diri sendiri }
%=============================================================================

%_____________________________________________________________________________
%=============================================================================
% 								BAGIAN XII
%=============================================================================
% Kata Pengantar: tempat anda menuliskan kata pengantar dan ucapan terima 
% kasih kepada yang telah membantu anda bla bla bla ....  
%=============================================================================
\prakata{\lipsum[1]}
%=============================================================================

%_____________________________________________________________________________
%=============================================================================
% 								BAGIAN XIII
%=============================================================================
% Tambahkan hyphen (pemenggalan kata) yang anda butuhkan di sini 
%=============================================================================
\hyphenation{ma-te-ma-ti-ka}
\hyphenation{fi-si-ka}
\hyphenation{tek-nik}
\hyphenation{in-for-ma-ti-ka}
%=============================================================================


%=============================================================================
