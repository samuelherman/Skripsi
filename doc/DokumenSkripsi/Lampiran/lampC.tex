\chapter{Wawancara}
\label{wawancara}

Hasil wawancara dengan beberapa dosen teknik informatika di fakultas teknik informasi dan sains.

\begin{enumerate}
\item Berikut hasil wawancara dengan seorang dosen Teknik Informatika bernama Pascal Alfadian pada tanggal 16 Juni 2015.\\\\
Pertanyaan : Bapak memiliki berapa anak wali?\\
Jawaban : 34\\\\
Pertanyaan : Kesulitan apa yang sering dialami dalam menjadi dosen wali?\\
Jawaban : Masalah administrasi seperti tanda tangan pada saat perwalian dan mahasiswa tidak memiliki sangsi adminstrasi jika tidak datang perwalian.\\\\\\
Pertanyaan : Apakah bapak sudah merasa optimal pada saat melakukan proses perwalian?\\
Jawaban : Masih bisa diperbaiki, selama ini saya melakukan survei ke 2-3 dosen untuk mengetahui informasi seorang mahasiswa.\\\\
Pertanyaan : Menurut bapak/ibu apa yang dapat membuat proses perwalian menjadi optimal?\\
Jawaban : Mendatangi dosen baik dosen yang memiliki relasi yang dekat dengan saya maupun yang tidak dekat dan keterbatasan waktu untuk membincangkan hal tersebut.\\\\
Pertanyaan : Apakah bapak merasa perlu untuk dibuatkan sebuah perangkat lunak yang membantu dalam pencatatan riwayat mahasiswa?\\
Jawaban : Ide yang menarik.\\\\
Pertanyaan : Fitur apa yang diharapkan?
Jawaban : Fitur yang mendukung agar saya menjadi efisien dalam menjadi dosen wali.\\

\item Berikut hasil wawancara dengan seorang dosen Teknik Informatika bernama Gede Karya pada tanggal 16 Juni 2015.\\\\
Pertanyaan : Bapak memiliki berapa anak wali?\\
Jawaban : 38\\
Pertanyaan : Kesulitan apa yang sering dialami dalam menjadi dosen wali?\\
Jawaban : Tidak ingat semua riwayat mahasiswa.\\\\
Pertanyaan : Apakah bapak sudah merasa optimal pada saat melakukan proses perwalian?\\
Jawaban : Ya, sudah tercapai.\\\\\\
Pertanyaan : Menurut bapak/ibu apa yang dapat membuat proses perwalian menjadi optimal?\\
Jawaban : Jadwal yang baik.\\\\
Pertanyaan : Apakah bapak merasa perlu untuk dibuatkan sebuah perangkat lunak yang membantu dalam pencatatan riwayat mahasiswa?\\
Jawaban : Ya, ide yang baik.\\\\
Pertanyaan : Fitur apa yang diharapkan?\\
Jawaban : Kebebasan dalam mencatat apa pun kemudian komentar positif dan komentar negatif.\\

\item Berikut hasil wawancara dengan seorang dosen Teknik Informatika bernama Cecilia Esti Nugraheni pada tanggal 16 Juni 2015.\\\\
Pertanyaan : Ibu memiliki berapa anak wali?\\
Jawaban : 30-40\\\\
Pertanyaan : Kesulitan apa yang sering dialami dalam menjadi dosen wali?\\
Jawaban : Tidak ada kesulitan untuk anak wali yang sejak awal saya pegang, namun ada kesulitan untuk mengenal mahasiswa yang mengambil topik skripsi saya, dan yang terakhir jika ada persoalan namun tidak cerita.\\\\
Pertanyaan : Apakah ibu sudah merasa optimal pada saat melakukan proses perwalian?\\
Jawaban : Belum, karena terbatas oleh waktu.\\\\
Pertanyaan : Menurut ibu apa yang dapat membuat proses perwalian menjadi optimal?\\
Jawaban : Menyediakan waktu khusus.\\\\
Pertanyaan : Apakah ibu merasa perlu untuk dibuatkan sebuah perangkat lunak yang membantu dalam pencatatan riwayat mahasiswa?\\
Jawaban : Saya setuju karena akan tertolong.\\
Pertanyaan : Fitur apa yang diharapkan?\\
Jawaban : Bisa bebas mencatat apa pun dan catatan khusus dosen wali.\\

\item Berikut hasil wawancara dengan seorang dosen Teknik Informatika bernama Joanna Helga pada tanggal 16 Juni 2015.\\\\
Pertanyaan : Ibu memiliki berapa anak wali?\\
Jawaban : 21 dari angkatan 2014 dan 5/6 mahasiswa yang sedang bimbingan.\\\\
Pertanyaan : Kesulitan apa yang sering dialami dalam menjadi dosen wali?\\
Jawaban : Ada mahasiswa yang tidak datang pada saat perwalian sehingga timbul pertanyaan mahasiswa tersebut akan datang atau tidak, dan apa perlu diingatkan tidak. Jadi saya perlu bertanya ke rekan dosen untuk mengetahui kondisi mahasiswa tersebut dikelas.\\\\
Pertanyaan : Apakah ibu sudah merasa optimal pada saat melakukan proses perwalian?\\
Jawaban : Tidak terlalu, karena sistem unpar aksesnya lama untuk mengetahui nilai mahasiswa kemudian anak wali malas cerita sehingga saya harus menkonfirmasi dengan dosen lain.\\\\
Pertanyaan : Menurut bapak/ibu apa yang dapat membuat proses perwalian menjadi optimal?\\
Jawaban : Kalau anak wali tidak memiliki masalah dan lebih mudah dalam mengakses permasalahan yang dimiliki seorang anak wali. Misalnya masalah keuangan dan kesulitan dalam kuliah programming.\\\\
Pertanyaan : Apakah ibu merasa perlu untuk dibuatkan sebuah perangkat lunak yang membantu dalam pencatatan riwayat mahasiswa?\\
Jawaban : Ya setuju.\\\\
Pertanyaan : Fitur apa yang diharapkan?\\
Jawaban : Catatan umum seperti telat kelas, sering tidak masuk kelas, dan sering tidak mengumpulkan tugas bukan ditulis supaya bisa direkap agar lebih kelihatan. Kemudian mudah dalam pengaksesan.\\

\item Berikut hasil wawancara dengan seorang dosen Teknik Informatika bernama Chandra Wijaya pada tanggal 17 Juni 2015.\\\\
Pertanyaan : Bapak memiliki berapa anak wali?\\
Jawaban : 40-50 \\\\
Pertanyaan : Kesulitan apa yang sering dialami dalam menjadi dosen wali?\\
Jawaban : Kadang harus disuruh-suruh untuk perwalian bahkan tidak datang.\\\\
Pertanyaan : Apakah bapak sudah merasa optimal pada saat melakukan proses perwalian?\\
Jawaban : Belum, karena harus mengingat {\it track record} setiap mahasiswa.\\\\
Pertanyaan : Menurut bapak apa yang dapat membuat proses perwalian menjadi optimal?\\
Jawaban : Setiap mahasiswa memiliki catatan, agar dapat dilihat memiliki masalah apa dan akibatnya apa.\\\\
Pertanyaan : Apakah bapak merasa perlu untuk dibuatkan sebuah perangkat lunak yang membantu dalam pencatatan riwayat mahasiswa?\\
Jawaban : Merasa sangat terbantu, apa lagi kalau dijaga oleh setiap dosen wali.\\\\
Pertanyaan : Fitur apa yang diharapkan?\\
Jawaban : Fitur yang ditawarkan sudah cukup.\\

\item Berikut hasil wawancara dengan seorang dosen Teknik Informatika bernama Veronica Sri Moertini pada tanggal 17 Juni 2015.\\\\
Pertanyaan : Ibu memiliki berapa anak wali?\\
Jawaban : 30 \\\\
Pertanyaan : Kesulitan apa yang sering dialami dalam menjadi dosen wali?\\
Jawaban : Sulit menghubungi mahasiswa, kemudian pada saat perwalian adanya menitipkan pada mahasiswa lain atau menghubungin melalui email atau whatsapp sehingga kurang intensif.\\\\
Pertanyaan : Apakah ibu sudah merasa optimal pada saat melakukan proses perwalian?\\
Jawaban : Sudah dengan adanya SIA, namun sisi diluar akademis tidak tercantum pada SIA seperti asal sekolah, bagaimana kondisi keluarga, dan tidak ada data diri.\\\\
Pertanyaan : Menurut ibu apa yang dapat membuat proses perwalian menjadi optimal?\\
Jawaban : Kebutuhan {\it cross check} dengan dosen lain misalnya kenapa nilai mahasiswa ini nilainya segini. Kemudian sulit untuk mengakses masalah yang dimiliki mahasiswa.\\\\
Pertanyaan : Apakah ibu merasa perlu untuk dibuatkan sebuah perangkat lunak yang membantu dalam pencatatan riwayat mahasiswa?\\
Jawaban : Setuju.\\\\
Pertanyaan : Fitur apa yang diharapkan?\\
Jawaban : Form mengakses masalah mahasiswa dan form kehadiran kelas yang mungkin dibutuhkan akses dari tata usaha atau asisten dosen.\\

%\item Berikut hasil wawancara dengan seorang dosen bernama Pascal pada tanggal 16 Juni 2015.\\\\
%Pertanyaan : Bapak/Ibu memiliki berapa anak wali?\\
%Pertanyaan : Kesulitan apa yang sering dialami dalam menjadi dosen wali?\\
%Pertanyaan : Kriteria apa saja yang biasanya diingat oleh bapak/ibu dari riwayat seorang mahasiswa?\\
%Pertanyaan : Apakah bapak/ibu sudah merasa optimal pada saat melakukan proses perwalian?\\
%Pertanyaan : Menurut bapak/ibu apa yang dapat membuat proses perwalian menjadi optimal?\\
%Pertanyaan : Apakah bapak/ibu merasa perlu untuk dibuatkan sebuah perangkat lunak yang membantu dalam pencatatan riwayat mahasiswa?\\
%Pertanyaan : Fitur apa yang diharapkan?\\
\end{enumerate}