\chapter{Pendahuluan}
\label{chap:pendahuluan}

\section{Latar Belakang}
\label{sec:latarbelakang}

Seorang mahasiswa pada umumnya dibimbing oleh dosen wali, dengan tugas-tugas tertentu, misalnya perwalian yang dilakukan setiap semester dan bimbingan dalam menyusun skripsi. Kasus di Falkutas Teknologi Informasi dan Sains UNPAR jurusan Teknik Informatika \cite{Ftis:2014}, jumlah mahasiswa 376 orang dengan 13 orang dosen pada semester genap 2013/2014, sehingga satu orang dosen menangani cukup banyak anak wali. Pada saat anak wali mengambil topik skripsi dari dosen lain, maka secara langsung anak wali tersebut pindah dosen wali sesuai dengan topik skripsi yang diambil. Kemudian ada dosen yang keluar masuk sehingga ada pergantian dosen wali. Maka dari itu hal-hal tersebut menimbulkan masalah bagi dosen wali untuk mengingat riwayat mahasiswa dalam jumlah yang terlalu besar.

Berdasarkan jabaran masalah diatas, maka solusi untuk membantu para dosen wali dalam mengingat riwayat mahasiswa adalah membuat sebuah sistem yang dapat mencatat dan membagikan riwayat mahasiswa. Dimana sistem hanya dapat diakses oleh dosen dengan menggunakan akun dosen yang diberikan oleh pihak UNPAR. Dalam sistem ini dosen dapat melakukan beberapa aksi. Pertama, dosen dapat melakukan login pada sistem dengan menggunakan akun dosen UNPAR. Kedua, dosen dapat melihat data mahasiswa yang telah terdaftar di basis data. Ketiga, dosen dapat mencari mahasiswa yang telat terdaftar di basis data berdasarkan npm. Keempat, dosen dapat melihat data informasi mahasiswa yang ingin dilihat. Kelima, dosen dapat mengubah data informasi mahasiswa. Keenam, dosen dapat membuat catatan masalah baru seorang mahasiswa. Ketujuh, dosen dapat melihat daftar masalah yang dimiliki seorang mahasiswa. Kedelapan, dosen dapat melihat histori yang dimiliki setiap mahasiswa (untuk aksi melihat, mengedit, membuat entri baru yang dilakukan dosen) dan dapat melihat setiap versi data informasi mahasiswa mulai dari pertama kali dibuat sampai yang terakhir. Kesembilan, dosen dapat membuat entri baru untuk mahasiswa yang belum tercatat pada basis data. 

Berdasarkan penjabaran solusi untuk menangani masalah, maka dari itu untuk membangun sistem usulan tersebut digunakan empat teknologi antara lain; Google OAuth, Markdown Syntax, StrapdownJS, Zurb Foundation. Google OAuth akan digunakan untuk fungsi login dalam mengautentikasi dan mengotorisasi pengguna yang menggunakan akun dosen UNPAR. Pengguna yang menggunakan akun selain akun dosen UNPAR tidak dapat mengakses sistem. Markdown Syntax akan digunakan untuk membuat informasi riwayat mahasiswa dengan format penulisan yang mudah dibaca dan ditulis oleh dosen sehingga penulisan pada sistem akan seragam. StrapdownJS akan digunakan pada sistem untuk mengkonversi format penulisan yang dibuat menggunakan Markdown Syntax menjadi tampilan HTML. Zurb Foundation akan digunakan untuk membuat tampilan antar muka menjadi responsif.

\section{Rumusan Masalah}
Berdasarkan latar belakang maka dapat dirumuskan permasalahan sebagai berikut:
\begin{itemize}
	\item Bagaimana mengautentikasi akun dosen UNPAR?
	\item Bagaimana membuat format untuk penulisan teks yang dapat digunakan oleh semua dosen?
	\item Bagaimana menampilkan teks dengan format yang telah dibuat ke halaman {\it website}?
	\item Bagaimana merancang antarmuka sistem menggunakan Zurb Foundation?
	\item Bagaimana mengukur tingkat keberhasilan sistem yang dibuat?
\end{itemize}

\section{Tujuan}
Berdasarkan rumusan masalah yang ditulis dalam sub bab 2, tujuan utama yang
ingin dicapai melalui penelitian ini adalah:
\begin{itemize}
	\item Mengautentikasi akun dosen UNPAR.
	\item Membuat format untuk penulisan teks yang dapat digunakan oleh semua dosen.
	\item Menampilkan teks dengan format yang telah dibuat ke halaman {\it website}.
	\item Merancang antarmuka sistem menggunakan Zurb Foundation.
	\item Mengukur tingkat keberhasilan sistem yang dibuat.
\end{itemize}

\section{Batasan Masalah}
Dalam penelitian ini ditetapkan batasan-batasan yang akan menjadi pedoman dalam
pelaksanaan penelitian:
\begin{itemize}
    \item Sistem membutuhkan akses internet.
	\item Sistem tidak terintegrasi dengan Sistem Informasi Akademik.
\end{itemize}

\section{Metodologi Penelitian}
Metodologi yang digunakan untuk menyusun penelitian:
\begin{itemize}
    \item Melakukan survei kebutuhan pengguna.
	\item Melakukan studi pustaka mengenai teknologi yang akan digunakan untuk
	membangun perangkat lunak.
	\item Menganalisis cara kerja teknologi yang akan digunakan untuk membangun
	perangkat lunak.
	\item Merancang perangkat lunak yang akan dibuat.
	\item Melakukan implementasi untuk perangkat lunak yang
	telah dirancang ke dalam {\it script} PHP.
	\item Melakukan pengujian perangkat lunak yang telah diimplementasikan.
	\item Melakukan pengambilan kesimpulan berdasarkan analisis dan pengujian yang telah dilakukan.
\end{itemize}

\section{Sistematika Pembahasan}
Sistematika pembahasan dalam penelitian ini adalah sebagai berikut:
\begin{itemize}
	\item Bab I Pendahuluan\\
	Bab ini menjelaskan latar belakang permasalahan, rumusan masalah, tujuan, batasan masalah, metodologi penelitian, dan sistematika pembahasan.
	\item Bab II Dasar Teori\\
	Bab ini menjelaskan teori-teori dasar mengenai Google OAuth, Markdown Syntax, StrapdownJS, dan Zurb Foundation yang menjadi refrensi utama dalam pelaksanaan penelitian.
	\item Bab III Analisis\\
	Bab ini berisi analisis mengenai kebutuhan pengguna, Google OAuth, Markdown Syntax, StrapdownJS, Zurb Foundation dan kebutuhan perangkat lunak yang akan digunakan pada penelitian ini.
	\item Bab IV Perancangan\\
	Bab ini berisi perancangan perangkat lunak yang akan dibuat.
	\item Bab V Implementasi dan Pengujian\\
	Bab ini berisi pengimplementasian dan pengujian perangkat lunak.
	\item Bab VI Kesimpulan dan Saran\\
	Bab ini berisi kesimpulan dari hasil penelitian dan saran untuk pengembangan lebih lanjut.
\end{itemize}