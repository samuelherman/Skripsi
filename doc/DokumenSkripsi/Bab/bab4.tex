\chapter{Perancangan}
\label{chap:perancangan}

Pada bab ini akan dijelaskan mengenai perancangan CAPTCHA yang akan dibuat. Mulai dari perancangan tampilan web yang digunakan, perancangan modul, dan perancangan diagram sekuens.

%\section{Perancangan Desain}
%\label{sec:perancangandesain}

%Pada tahap perancangan desain ada dua desain atau rancangan yang dibuat yaitu {\it flowchart view} dan {\it flowchart system}.

%\subsection{Flowchart View}

%Pada perancangan flowchart view, dibuat untuk mengetahui semua elemen yang ada pada halaman awal atau menu. Halaman awal atau menu memiliki beberapa elemen antara lain : Gambar CAPTCHA, text area, tombol submit, notifikasi, hasil OCR, tombol history, tabel history. Gambar dapat dilihat pada Gambar~\ref{fig:fcv}.

%\begin{figure}[H]
%\centering
%\includegraphics[scale=0.5]{Gambar/flowchartview.jpg}
%\caption[Flowchart View]{Flowchart View} 
%\label{fig:fcv}
%\end{figure}

%\subsection{Flowchart Sistem}

%Selain flowchart view, dibuat pula perancangan sistem secara menyeluruh dengan menggunakan flowchart sistem. Flowchart sistem ini menggambarkan tahap-tahap proses yang ada pada sistem dari awal hingga akhir. Untuk flowchart sistem dapat dilihat pada Gambar~\ref{fig:fcs}.

%Tahaptahap proses yang ada pada sistem:
%\begin{enumerate}
%\item
%Proses pertama dimulai dari membangkitkan CAPTCHA yang dilakukan oleh sistem secara otomatis.

%\item
%Proses kedua user menginputkan solusi dari CAPTCHA yang telah dibangkitkan.

%\item
%Proses ketiga hasil solusi dari user akan keluar beserta hasil solusi dari OCR akan tersimpan.

%\item
%Proses keempat hasil akan ditampilkan untuk dilihat oleh user dan terlihat perbandingan hasil solusi CAPTCHA dari segi manusia dan komputer. 
%\end{enumerate}

%Tahap proses pertama dimulai dari membangkitkan CAPTCHA lalu user menginputkan solusi dari CAPTCHA yang telah dibangkitkan. Kemudian hasil solusi akan keluar beserta hasil solusi dari OCR, lalu terlihat hasil perbandingannya dan selesai. 

%\begin{figure}[H]
%\centering
%\includegraphics[scale=0.5]{Gambar/flowchartsistem.jpg}
%\caption[Flowchart Sistem]{Flowchart Sistem} 
%\label{fig:fcs}
%\end{figure}

\section{Perancangan Tampilan Web Yang Digunakan}
\label{sec:perancanganantarmuka}

Perancangan tampilan web yang akan dibuat untuk mengimplementasikan CAPTCHA dapat dilihat pada Gambar~\ref{fig:antarmuka}.

\begin{figure}[H]
\centering
\includegraphics[scale=0.5]{Gambar/antarmuka.jpg}
\caption[Desain Antarmuka]{Desain Antarmuka} 
\label{fig:antarmuka}
\end{figure}

Keterangan :
\begin{enumerate}
\item
Bagian ini untuk menentukan panjang karakter CAPTCHA yang akan dibangkitkan. User memilih sendiri untuk panjang karakter CAPTCHA.

\item
Bagian ini merupakan area untuk menampilkan gambar CAPTCHA yang telah dibangkitkan.

\item
Bagian ini merupakan tempat memasukan jawaban CAPTCHA dari user.

\item
Bagian ini merupakan tempat menampilkan jawaban CAPTCHA dari OCR.

\item
Bagian ini untuk mengsubmit jawaban user.

\item
Bagian ini merupakan tabel yang berisi riwayat penggunaan perangakat lunak.

\end{enumerate}

\section{Perancangan Modul}
\label{sec:perancanganmodul}

Perancangan modul untuk CAPTCHA yang akan dibuat dapat dilihat pada Gambar~\ref{fig:modul}.

\begin{figure}[H]
\centering
\includegraphics[scale=0.75]{Gambar/modul.jpg}
\caption[Struktur Modul]{Struktur Modul} 
\label{fig:modul}
\end{figure}

\begin{enumerate}
\item
Modul Generate CAPTCHA\\
Input: -\\
Output: Gambar CAPTCHA\\
Deskripsi: Gambar CAPTCHA akan dibangkitkan secara otomatis.

\item
Modul Input Solusi\\
Input: Gambar CAPTCHA\\
Output: Solusi dari user\\
Deskripsi:Gambar CAPTCHA dilihat oleh user dan user memberikan solusi guna mengukur kualitas CAPTCHA yang dibuat dari segi manusia.

\item
Modul OCR\\
Input: Gambar CAPTCHA\\
Output: Solusi dari OCR\\
Deskripsi: Gambar CAPTCHA diambil untuk diproses oleh OCR guna mengukur kualitas CAPTCHA yang dibuat dari segi komputer.

\item
Modul Hasil\\
Input: Hasil dari inputan user dan hasil dari OCR\\
Output: Perbandingan hasil dari user dan OCR\\
Deskripsi: Hasil ini untuk melihat perbandingan hasil solusi CAPTCHA.

\end{enumerate}

%\section{Deskripsi Modul}
%\label{sec:deskripsimodul}

\section{Diagram Sekuens}
\label{sec:diagramsekuens}

Pembuatan diagram sekuens mengacu pada Gambar~\ref{fig:usecase}. Diagram sekuens dapat dilihat pada Gambar \ref{fig:ds}.

\begin{figure}[H]
\centering
\includegraphics[scale=0.5]{Gambar/diagramsekuens.jpg}
\caption[Diagram Sekuens]{Diagram Sekuens} 
\label{fig:ds}
\end{figure}
