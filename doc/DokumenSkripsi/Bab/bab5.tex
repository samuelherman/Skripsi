\chapter{Implementasi dan Pengujian Perangkat Lunak}
\label{chap:implementasidanpengujian}

Bab ini terdiri atas dua bagian, yaitu Implementasi Perangkat Lunak dan Pengujian Perangkat Lunak. Bagian implementasi berisi penjelasan lingkungan pengembangan perangkat lunak. Sedangkan bagian pengujian berisi hasil pengujian terhadap perangkat lunak yang telah dibangun.

\section{Implementasi Perangkat Lunak}
\label{sec:implementasiperangkatlunak}

Pada bagian ini akan dibahas hasil implementasi perangkat lunak yang telah dibangun. Subbab ini terdiri atas tiga bagian, yaitu lingkungan perangkat keras, lingkungan perangkat lunak, dan hasil implementasi perangkat lunak.

\subsection{Lingkungan Implementasi Perangkat Keras}
\label{sec:lingkunganimplementasiperangkatkeras}

Dalam membangun perangkat lunak ini digunakan spesifikasi perangkat keras sebagai berikut:

\begin{enumerate}
\item[(a)] Processor: AMD E-350 @1.6GHz
\item[(b)] RAM: 2.00 GB DDR3
\item[(c)] Harddisk: 320 GBThird item
\item[(d)] VGA: ATI Radeon HD 6310 1GB
\end{enumerate}

\subsection{Lingkungan Implementasi Perangkat Lunak}
\label{sec:lingkunganimplementasiperangkatlunak}

Dalam membangun perangkat lunak ini digunakan spesifikasi perangkat lunak sebagai berikut:

\begin{enumerate}
\item[(a)] Sistem Operasi: Windows 7 Professional 64-bit
\item[(b)] Bahasa Pemrograman: PHP v5.5.11
\item[(c)] IDE: XAMPP v1.8.3
\item[(d)] Library PHP: GD v2.1.0
\end{enumerate}

\subsection{Hasil Implementasi Perangkat Lunak}
\label{sec:hasilimplementasi}

Hasil implementasi dari perangkat lunak ini. Kode program perangkat lunak ditulis berdasarkan perancangan yang telah dibahas pada bab~\ref{chap:perancangan}. Kode program untuk membangkitkan CAPTCHA menggunakan PHP dengan library GD dan library tesseract OCR.

\section{Pengujian Perangkat Lunak}
\label{sec:pengujianperangkatlunak}

Pada bagian ini akan dibahas mengenai pengujian yang akan dilakukan terhadap perangkat lunak. Pengujian tersebut terdiri dari dua bagian yaitu pengujian fungsional dan pengujian eksperimental. Pengujian fungsional bertujuan untuk memastikan bahwa seluruh fungsi yang dibangun pada perangkat lunak berjalan sesuai dengan yang direncanakan. Pembangunan perangkat lunak yang dibangun ini dilakukan dengan menggunakan metode berorientasi objek, maka pengujian akan dilakukan melalui skenario program. Sedangkan pengujian eksperimental bertujuan untuk menguji eksepsi-eksepsi yang terdapat pada perangkat lunak.

\subsection{Lingkungan Pengujian Perangkat Keras}
\label{sec:lingkunganpengujianperangkatkeras}

Dalam pengujian perangkat lunak ini digunakan spesifikasi perangkat keras sebagai berikut:

\begin{enumerate}
\item[(a)] Processor: AMD E-350 @1.6GHz
\item[(b)] RAM: 2.00 GB DDR3
\item[(c)] Harddisk: 320 GBThird item
\item[(d)] VGA: ATI Radeon HD 6310 1GB
\end{enumerate}

\subsection{Lingkungan Pengujian Perangkat Lunak}
\label{sec:lingkunganpengujianperangkatlunak}

Dalam pengujian perangkat lunak ini digunakan spesifikasi perangkat lunak sebagai berikut:

\begin{enumerate}
\item[(a)] Sistem Operasi: Windows 7 Professional 64-bit
\item[(b)] Bahasa Pemrograman: PHP v5.5.11
\item[(c)] IDE: XAMPP v1.8.3
\item[(d)] Library PHP: GD v2.1.0
\end{enumerate}