\chapter{Kesimpulan dan Saran}
\label{chap:kesimpulandansaran}

Pada bab ini akan diberikan kesimpulan yang didapat dari proses perancangan dan
pengujian perangkat lunak yang dibangun, juga saran-saran untuk penelitian ini
jika ingin dikembangkan di kemudian hari.

\section{Kesimpulan}
\label{sec:kesimpulan}
Kesimpulan yang didapat dari pembangunan perangkat lunak Sistem Perekam dan Berbagi Riwayat Mahasiswa (SPBRM) anatara lain :
\begin{enumerate}[(1)]
  \item Autentikasi dan otorisasi akun dosen UNPAR dapat menggunakan Google OAuth. Perangkat lunak menggunakan Google OAuth pada fungsi login. 
  \item Format penulisan yang seragam dapat dibuat dengan menggunakan Markdown. Perangkat lunak menggunakan Markdown untuk menuliskan riwayat mahasiswa pada fitur mengedit info mahasiswa dan fitur membuat entri baru.
  \item Menampilkan teks dengan format Markdown ke HTML dapat menggunakan StrapdownJS. Perangkat lunak menggunakan StrapdownJS untuk menampilkan riwayat mahasiswa yang terdapat fitur melihat info mahasiswa dan fitur melihat versi ini pada histori.
  \item Penggunaan Zurb Foundation berhasil membuat tampilan antarmuka perangkat
  lunak Sistem Perekam dan Berbagi Riwayat Mahasiswa (SPBRM) menjadi responsif.
  \item Pada tahap pengujian dapat disimpulkan bahwa perangkat lunak Sistem
  Perekam dan Berbagi Riwayat Mahasiswa (SPBRM) sudah berjalan dengan baik dan memberikan
  keluaran sesuai yang diharapkan pengguna.
\end{enumerate}

\section{Saran}
\label{sec:saran}
Saran yang dapat diberikan untuk perbaikan dan pengembangan Sistem Perekam dan Berbagi Riwayat Mahasiswa (SPBRM) antara lain :
\begin{enumerate}[(1)]
  \item Tampilan antarmuka masih kurang menarik terutama untuk tata letak tombol 'Simpan' pada edit.php dan new.php, untuk pengembangan dapat menggunakan navigasi dan {\it plugins} yang dimiliki Zurb Foundation.
  \item Pada fungsi membuat entri baru, untuk pengembangan data npm terhubung dengan data nama. Jadi cukup dengan memasukkan npm seorang mahasiswa, lalu pada bagian nama otomatis mengeluarkan nama mahasiswa dengan npm tersebut.
  \item Pada fungsi histori, untuk pengembangan dapat ditambahkan keterangan yang
  membandingkan versi baru dan versi lama. Jadi pengguna mengetahui bagian mana
  yang dihapus dan bagian mana yang ditambah atau dirubah.
\end{enumerate}