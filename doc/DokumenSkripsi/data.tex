%_____________________________________________________________________________
%=============================================================================
% data.tex v6 (13-04-2015) \ldots dibuat oleh Lionov - Informatika FTIS UNPAR
%
% Perubahan pada versi 6 (13-04-2015)
% - Perubahan untuk data-data ``template" menjadi lebih generik dan menggunakan
%	tanda << dan >>
%
% Perubahan pada versi sebelumnya
% 	versi 5 (10-11-2013)
% 	- Perbaikan pada memasukkan bab : tidak perlu menuliskan apapun untuk 
%	  memasukkan seluruh bab (bagian V)
% 	- Perbaikan pada memasukkan lampiran : tidak perlu menuliskan apapun untuk
%	  memasukkan seluruh lampiran atau -1 jika tidak memasukkan apapun
%	versi 4 (21-10-2012)
%	- Data dosen dipindah ke dosen.tex agar jika ada perubahan/update data dosen
%   mahasiswa tidak perlu mengubah data.tex
%	- Perubahan pada keterangan dosen	
% 	versi 3 (06-08-2012)
% 	- Perubahan pada beberapa keterangan 
% 	versi 2 (09-07-2012):
% 	- Menambahkan data judul dalam bahasa inggris
% 	- Membuat bagian khusus untuk judul (bagian VIII)
% 	- Perbaikan pada gelar dosen
%_____________________________________________________________________________
%=============================================================================
% 								BAGIAN -
%=============================================================================
% Ini adalah file data (data.tex)
% Masukkan ke dalam file ini, data-data yang diperlukan oleh template ini
% Cara memasukkan data dijelaskan di setiap bagian
% Data yang WAJIB dan HARUS diisi dengan baik dan benar adalah SELURUHNYA !!
% Hilangkan tanda << dan >> jika anda menemukannya
%=============================================================================
%_____________________________________________________________________________
%=============================================================================
% 								BAGIAN I
%=============================================================================
% Tambahkan package2 lain yang anda butuhkan di sini
%=============================================================================
\usepackage{amsmath}
\usepackage{booktabs}
\usepackage{enumerate}
\usepackage{longtable}
\usepackage{multirow}
\usepackage{amssymb}
\usepackage{bbding}
\usepackage[table]{xcolor}
\usepackage{tikz}
\usepackage{titlesec}\usepackage[table]{xcolor}
\usepackage{longtable}
\usepackage{amsmath}
%=============================================================================

%_____________________________________________________________________________
%=============================================================================
% 								BAGIAN II
%=============================================================================
% Mode dokumen: menetukan halaman depan dari dokumen, apakah harus mengandung 
% prakata/pernyataan/abstrak dll (termasuk daftar gambar/tabel/isi) ?
% - kosong : tidak ada halaman depan sama sekali (untuk dokumen yang 
%            dipergunakan pada proses bimbingan)
% - cover : cover saja tanpa daftar isi, gambar dan tabel
% - sidang : cover, daftar isi, gambar, tabel (IT: UTS-UAS Seminar 
%			 dan UTS TA)
% - sidang_akhir : mode sidang + abstrak + abstract
% - final : seluruh halaman awal dokumen (untuk cetak final)
% Jika tidak ingin mencetak daftar tabel/gambar (misalkan karena tidak ada 
% isinya), edit manual di baris 439 dan 440 pada file main.tex
%=============================================================================
%\mode{kosong}
%\mode{cover}
%\mode{sidang}
%\mode{sidang_akhir}
\mode{final} 
%=============================================================================

%_____________________________________________________________________________
%=============================================================================
% 								BAGIAN III
%=============================================================================
% Line numbering: penomoran setiap baris, otomatis di-reset setiap berganti
% halaman
% - yes: setiap baris diberi nomor
% - no : baris tidak diberi nomor, otomatis untuk mode final
%=============================================================================
\linenumber{yes}
%=============================================================================

%_____________________________________________________________________________
%=============================================================================
% 								BAGIAN IV
%=============================================================================
% Linespacing: jarak antara baris 
% - single: opsi yang disediakan untuk bimbingan, jika pembimbing tidak
%            keberatan (untuk menghemat kertas)
% - onehalf: default dan wajib (dan otomatis) jika ingin mencetak dokumen
%            final/untuk sidang.
% - double : jarak yang lebih lebar lagi, jika pembimbing berniat memberi 
%            catatan yg banyak di antara baris (dianjurkan untuk bimbingan)
%=============================================================================
\linespacing{single}
% \linespacing{onehalf}
%\linespacing{double}
%=============================================================================

%_____________________________________________________________________________
%=============================================================================
% 								BAGIAN V
%=============================================================================
% Bab yang akan dicetak: isi dengan angka 1,2,3 s.d 9, sehingga bisa digunakan
% untuk mencetak hanya 1 atau beberapa bab saja
% Jika lebih dari 1 bab, pisahkan dengan ',', bab akan dicetak terurut sesuai 
% urutan bab.
% Untuk mencetak seluruh bab, kosongkan parameter (i.e. \bab{ })  
% Catatan: Jika ingin menambahkan bab ke-10 dan seterusnya, harus dilakukan 
% secara manual
%=============================================================================
\bab{ }
%=============================================================================

%_____________________________________________________________________________
%=============================================================================
% 								BAGIAN VI
%=============================================================================
% Lampiran yang akan dicetak: isi dengan huruf A,B,C s.d I, sehingga bisa 
% digunakan untuk mencetak hanya 1 atau beberapa lampiran saja
% Jika lebih dari 1 lampiran, pisahkan dengan ',', lampiran akan dicetak 
% terurut sesuai urutan lampiran
% Jika tidak ingin mencetak lampiran apapun, isi dengan -1 (i.e. \lampiran{-1})
% Untuk mencetak seluruh mapiran, kosongkan parameter (i.e. \lampiran{ })  
% Catatan: Jika ingin menambahkan lampiran ke-J dan seterusnya, harus 
% dilakukan secara manual
%=============================================================================
\lampiran{ }
%=============================================================================

%_____________________________________________________________________________
%=============================================================================
% 								BAGIAN VII
%=============================================================================
% Data diri dan skripsi/tugas akhir
% - namanpm: Nama dan NPM anda, penggunaan huruf besar untuk nama harus benar
%			 dan gunakan 10 digit npm UNPAR, PASTIKAN BAHWA BENAR !!!
%			 (e.g. \namanpm{Jane Doe}{1992710001}
% - judul : Dalam bahasa Indonesia, perhatikan penggunaan huruf besar, judul
%			tidak menggunakan huruf besar seluruhnya !!! 
% - tanggal : isi dengan {tangga}{bulan}{tahun} dalam angka numerik, jangan 
%			  menuliskan kata (e.g. AGUSTUS) dalam isian bulan
%			  Tanggal ini adalah tanggal dimana anda akan melaksanakan sidang 
%			  ujian akhir skripsi/tugas akhir
% - pembimbing: isi dengan pembimbing anda, lihat daftar dosen di file dosen.tex
%				jika pembimbing hanya 1, kosongkan parameter kedua 
%				(e.g. \pembimbing{\JND}{  } ) , \JND adalah kode dosen
% - penguji : isi dengan para penguji anda, lihat daftar dosen di file dosen.tex
%				(e.g. \penguji{\JHD}{\JCD} ) , \JND dan \JCD adalah kode dosen
%
%=============================================================================
\namanpm{Samuel Herman}{2010730013}	%hilangkan tanda << & >>
\tanggal{27}{5}{2015}			%hilangkan tanda << & >>
\pembimbing{\CEN}{}     
%Lihat singkatan pembimbing anda di file dosen.tex, hilangkan tanda << & >>
\penguji{\LNV}{\VSM} 		
%Lihat singkatan penguji anda di file dosen.tex, hilangkan tanda << & >>
%=============================================================================

%_____________________________________________________________________________
%=============================================================================
% 								BAGIAN VIII
%=============================================================================
% Judul dan title : judul bhs indonesia dan inggris
% - judulINA: judul dalam bahasa indonesia
% - judulENG: title in english
% PERHATIAN: - langsung mulai setelah '{' awal, jangan mulai menulis di baris 
%			   bawahnya
%			 - Gunakan \texorpdfstring{\\}{} untuk pindah ke baris baru
%			 - Judul TIDAK ditulis dengan menggunakan huruf besar seluruhnya !!
%			 - Gunakan perintah \texorpdfstring{\\}{} untuk baris baru
%=============================================================================

\judulINA{Sistem Perekam dan Berbagi Riwayat Mahasiswa untuk Dosen}

\judulENG{Student History Recording and Sharing System for Lecturers}

%_____________________________________________________________________________
%=============================================================================
% 								BAGIAN IX
%=============================================================================
% Abstrak dan abstract : abstrak bhs indonesia dan inggris
% - abstrakINA: abstrak bahasa indonesia
% - abstrakENG: abstract in english
% PERHATIAN: langsung mulai setelah '{' awal, jangan mulai menulis di baris 
%			 bawahnya
%=============================================================================

\abstrakINA{Fakultas Teknologi Informasi dan Sains memiliki 14 dosen Teknik Informatika yang memiliki tugas untuk berinteraksi dengan banyak mahasiswa. Hal tersebut menimbulkan masalah bagi dosen dalam mengingat setiap riwayat mahasiswa yang berinteraksi dengannya. Untuk masalah tersebut, dibuatkan sebuah sistem yang mencatat setiap riwayat mahasiswa dan membagikan setiap riwayat mahsiswa ke semua dosen. Pada skripsi ini, dibahas pembangunan Sistem Perekam dan Berbagi Riwayat Mahasiswa (SPBRM) yang menggunakan 4 teknologi; Google OAuth, Markdown, StrapdownJS, Zurb Foundation.

SPBRM menggunakan Google OAuth untuk mengautentikasi dan mengotorisasi pengguna yang akan menggunakan akun dosen UNPAR. SPBRM menggunakan sintakis Markdown untuk membuat format penulisan teks menjadi seragam. SPBRM menggunakan StrapdownJS untuk mengkonversi teks dengan format penulisan yang telah dibuat menggunakan Markdown menjadi tampilan HTML. SPBRM menggunakan Zurb Foundation untuk membuat tampilan antarmuka yang responsif.\\\\\\}  

\abstrakENG{Faculty Information Technology and Science has 14 lecturers that are required to interact with the students. This poses a problem for the lecturer remembers every student in history. So it's good to make a system that records every history student and share any history of student to all professors. In this book, discussed the construction of Student History Recording and Sharing System (SPBRM) that uses 4 technology; Google OAuth, Markdown, StrapdownJS, Zurb Foundation.

SPBRM use Google OAuth to authenticate and authorize the user accounts that will be using lecturer UNPAR account. SPBRM uses the Markdown formatting to make writing text into uniform. SPBRM use StrapdownJS to convert text with writing format that has been created using Markdown HTML into view. SPBRM using the Zurb Foundation to create the look of the interface is responsive.\\\\\\} 

%=============================================================================

%_____________________________________________________________________________
%=============================================================================
% 								BAGIAN X
%=============================================================================
% Kata-kata kunci dan keywords : diletakkan di bawah abstrak (ina dan eng)
% - kunciINA: kata-kata kunci dalam bahasa indonesia
% - kunciENG: keywords in english
%=============================================================================
\kunciINA{Google OAuth, Markdown, StrapdownJS, Zurb Foundation}

\kunciENG{Google OAuth, Markdown, StrapdownJS, Zurb Foundation}
%=============================================================================

%_____________________________________________________________________________
%=============================================================================
% 								BAGIAN XI
%=============================================================================
% Persembahan : kepada siapa anda mempersembahkan skripsi ini ...
%=============================================================================
\untuk{Dipersembahkan untuk diri sendiri, keluarga, teman-teman, dan
semua orang yang berperan dalam pembuatan skripsi ini.}
%=============================================================================

%_____________________________________________________________________________
%=============================================================================
% 								BAGIAN XII
%=============================================================================
% Kata Pengantar: tempat anda menuliskan kata pengantar dan ucapan terima 
% kasih kepada yang telah membantu anda bla bla bla ....  
%=============================================================================
\prakata{Terima kasih kepada Tuhan Yesus Kristus yang telah memberikan kesempatan kepada penulis untuk dapat menyelesaikan skripsi dengan judul ''Sistem Perekan dan Berbagi Riwayat Mahasiswa untuk Dosen''. Banyak kesulitan yang dihadapi dan berbagai masalah yang menjadikan beban pikiran selama proses pengerjaan skripsi ini. Skripsi ini dibuat sebagai salah satu syarat kelulusan di Universitas Katolik Parahyangan. Penulis juga ingin mengucapkan terima kasih kepada beberapa pihak yang telah membantu baik secara langsung maupun tidak langsung dalam proses pengerjaan skripsi ini, karena tanpa pihak-pihak tersebut skripsi ini mungkin tidak akan pernah selesai. Beberapa pihak tersebut adalah:
\begin{enumerate}
\item Ayah (Herman Limandjaja), Ibu (Emmi Muljati), Adik (Andreas Herman), dan seluruh keluarga yang telah memberikan doa dan dukungan.
\item Bapak Pascal Alfadian, M.Com. sebagai dosen pembimbing yang selalu sabar dan setia dalam memberikan arahan, kritik, dan dukungan selama proses penyusunan skripsi.
\item Bapak Lionov, M.Sc. dan Ibu Dr. Veronica Sri Moertini, Ir., MT. sebagai dosen penguji yang telah memberikan koreksi, kritik, dan saran agar skripsi ini menjadi lebih baik.
\item Seluruh dosen di Jurusan Teknik Informatika Fakultas Teknologi Informasi dan Sains, Universitas Katolik Parahyangan.
\item Seluruh anak Archer Youth Community yang selalu memberikan doa dan dukungan.
\item Billy Sebastian, Yohanes Subrata, Kenneth Natanael, Liviana Devi, dan Nerisa Arviana sebagai sahabat SBYjK.
\item Andreas Haryawan, Andri Agustian, Dominikus, Edwin Herawanputra, Grady Ireneus, Hans Wirya, Henry Setiadi, Kevin PL, Reyner Subrata, Rico Fransisco, Stephanus Jeffry, dan Steven Tjiardy sebagai teman bermain dan rekan seperjuangan di Unpar.
\item Seluruh teman-teman IT Unpar 2010.
\item Pihak-pihak yang membantu dan tidak dapat penulis sebutkan satu persatu.
\end{enumerate}
Akhir kata penulis berharap agar skripsi ini dapat memberikan manfaat kepada orang-orang yang membaca. Penulis juga memohon maaf apabila terdapat kesalahan penulisan nama atau kata-kata yang kurang berkenan. Tuhan Yesus memberkati. Amin.}
%=============================================================================

%_____________________________________________________________________________
%=============================================================================
% 								BAGIAN XIII
%=============================================================================
% Tambahkan hyphen (pemenggalan kata) yang anda butuhkan di sini 
%=============================================================================
\hyphenation{ma-te-ma-ti-ka}
\hyphenation{fi-si-ka}
\hyphenation{tek-nik}
\hyphenation{in-for-ma-ti-ka}
\hyphenation{me-nim-bul-kan}
\hyphenation{pe-rang-kat}
\hyphenation{skrip-si}
\hyphenation{meng-a-rah-kan}
\hyphenation{di-ja-di-kan}
\hyphenation{men-ja-lan-kan}
\hyphenation{}
\hyphenation{}
\hyphenation{}
\hyphenation{}
%=============================================================================


%=============================================================================
